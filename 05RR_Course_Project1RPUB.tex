\documentclass[]{article}
\usepackage{lmodern}
\usepackage{amssymb,amsmath}
\usepackage{ifxetex,ifluatex}
\usepackage{fixltx2e} % provides \textsubscript
\ifnum 0\ifxetex 1\fi\ifluatex 1\fi=0 % if pdftex
  \usepackage[T1]{fontenc}
  \usepackage[utf8]{inputenc}
\else % if luatex or xelatex
  \ifxetex
    \usepackage{mathspec}
  \else
    \usepackage{fontspec}
  \fi
  \defaultfontfeatures{Ligatures=TeX,Scale=MatchLowercase}
\fi
% use upquote if available, for straight quotes in verbatim environments
\IfFileExists{upquote.sty}{\usepackage{upquote}}{}
% use microtype if available
\IfFileExists{microtype.sty}{%
\usepackage{microtype}
\UseMicrotypeSet[protrusion]{basicmath} % disable protrusion for tt fonts
}{}
\usepackage[margin=1in]{geometry}
\usepackage{hyperref}
\hypersetup{unicode=true,
            pdftitle={Reproducible Research Project 1},
            pdfauthor={Ravinendra Pratap},
            pdfborder={0 0 0},
            breaklinks=true}
\urlstyle{same}  % don't use monospace font for urls
\usepackage{color}
\usepackage{fancyvrb}
\newcommand{\VerbBar}{|}
\newcommand{\VERB}{\Verb[commandchars=\\\{\}]}
\DefineVerbatimEnvironment{Highlighting}{Verbatim}{commandchars=\\\{\}}
% Add ',fontsize=\small' for more characters per line
\usepackage{framed}
\definecolor{shadecolor}{RGB}{248,248,248}
\newenvironment{Shaded}{\begin{snugshade}}{\end{snugshade}}
\newcommand{\KeywordTok}[1]{\textcolor[rgb]{0.13,0.29,0.53}{\textbf{#1}}}
\newcommand{\DataTypeTok}[1]{\textcolor[rgb]{0.13,0.29,0.53}{#1}}
\newcommand{\DecValTok}[1]{\textcolor[rgb]{0.00,0.00,0.81}{#1}}
\newcommand{\BaseNTok}[1]{\textcolor[rgb]{0.00,0.00,0.81}{#1}}
\newcommand{\FloatTok}[1]{\textcolor[rgb]{0.00,0.00,0.81}{#1}}
\newcommand{\ConstantTok}[1]{\textcolor[rgb]{0.00,0.00,0.00}{#1}}
\newcommand{\CharTok}[1]{\textcolor[rgb]{0.31,0.60,0.02}{#1}}
\newcommand{\SpecialCharTok}[1]{\textcolor[rgb]{0.00,0.00,0.00}{#1}}
\newcommand{\StringTok}[1]{\textcolor[rgb]{0.31,0.60,0.02}{#1}}
\newcommand{\VerbatimStringTok}[1]{\textcolor[rgb]{0.31,0.60,0.02}{#1}}
\newcommand{\SpecialStringTok}[1]{\textcolor[rgb]{0.31,0.60,0.02}{#1}}
\newcommand{\ImportTok}[1]{#1}
\newcommand{\CommentTok}[1]{\textcolor[rgb]{0.56,0.35,0.01}{\textit{#1}}}
\newcommand{\DocumentationTok}[1]{\textcolor[rgb]{0.56,0.35,0.01}{\textbf{\textit{#1}}}}
\newcommand{\AnnotationTok}[1]{\textcolor[rgb]{0.56,0.35,0.01}{\textbf{\textit{#1}}}}
\newcommand{\CommentVarTok}[1]{\textcolor[rgb]{0.56,0.35,0.01}{\textbf{\textit{#1}}}}
\newcommand{\OtherTok}[1]{\textcolor[rgb]{0.56,0.35,0.01}{#1}}
\newcommand{\FunctionTok}[1]{\textcolor[rgb]{0.00,0.00,0.00}{#1}}
\newcommand{\VariableTok}[1]{\textcolor[rgb]{0.00,0.00,0.00}{#1}}
\newcommand{\ControlFlowTok}[1]{\textcolor[rgb]{0.13,0.29,0.53}{\textbf{#1}}}
\newcommand{\OperatorTok}[1]{\textcolor[rgb]{0.81,0.36,0.00}{\textbf{#1}}}
\newcommand{\BuiltInTok}[1]{#1}
\newcommand{\ExtensionTok}[1]{#1}
\newcommand{\PreprocessorTok}[1]{\textcolor[rgb]{0.56,0.35,0.01}{\textit{#1}}}
\newcommand{\AttributeTok}[1]{\textcolor[rgb]{0.77,0.63,0.00}{#1}}
\newcommand{\RegionMarkerTok}[1]{#1}
\newcommand{\InformationTok}[1]{\textcolor[rgb]{0.56,0.35,0.01}{\textbf{\textit{#1}}}}
\newcommand{\WarningTok}[1]{\textcolor[rgb]{0.56,0.35,0.01}{\textbf{\textit{#1}}}}
\newcommand{\AlertTok}[1]{\textcolor[rgb]{0.94,0.16,0.16}{#1}}
\newcommand{\ErrorTok}[1]{\textcolor[rgb]{0.64,0.00,0.00}{\textbf{#1}}}
\newcommand{\NormalTok}[1]{#1}
\usepackage{graphicx,grffile}
\makeatletter
\def\maxwidth{\ifdim\Gin@nat@width>\linewidth\linewidth\else\Gin@nat@width\fi}
\def\maxheight{\ifdim\Gin@nat@height>\textheight\textheight\else\Gin@nat@height\fi}
\makeatother
% Scale images if necessary, so that they will not overflow the page
% margins by default, and it is still possible to overwrite the defaults
% using explicit options in \includegraphics[width, height, ...]{}
\setkeys{Gin}{width=\maxwidth,height=\maxheight,keepaspectratio}
\IfFileExists{parskip.sty}{%
\usepackage{parskip}
}{% else
\setlength{\parindent}{0pt}
\setlength{\parskip}{6pt plus 2pt minus 1pt}
}
\setlength{\emergencystretch}{3em}  % prevent overfull lines
\providecommand{\tightlist}{%
  \setlength{\itemsep}{0pt}\setlength{\parskip}{0pt}}
\setcounter{secnumdepth}{0}
% Redefines (sub)paragraphs to behave more like sections
\ifx\paragraph\undefined\else
\let\oldparagraph\paragraph
\renewcommand{\paragraph}[1]{\oldparagraph{#1}\mbox{}}
\fi
\ifx\subparagraph\undefined\else
\let\oldsubparagraph\subparagraph
\renewcommand{\subparagraph}[1]{\oldsubparagraph{#1}\mbox{}}
\fi

%%% Use protect on footnotes to avoid problems with footnotes in titles
\let\rmarkdownfootnote\footnote%
\def\footnote{\protect\rmarkdownfootnote}

%%% Change title format to be more compact
\usepackage{titling}

% Create subtitle command for use in maketitle
\newcommand{\subtitle}[1]{
  \posttitle{
    \begin{center}\large#1\end{center}
    }
}

\setlength{\droptitle}{-2em}

  \title{Reproducible Research Project 1}
    \pretitle{\vspace{\droptitle}\centering\huge}
  \posttitle{\par}
    \author{Ravinendra Pratap}
    \preauthor{\centering\large\emph}
  \postauthor{\par}
      \predate{\centering\large\emph}
  \postdate{\par}
    \date{10 December 2018}


\begin{document}
\maketitle

\subsection{R Markdown}\label{r-markdown}

\section{Reproducible Research: Peer Assessment
1}\label{reproducible-research-peer-assessment-1}

\subsection{Loading and preprocessing the
data}\label{loading-and-preprocessing-the-data}

\begin{Shaded}
\begin{Highlighting}[]
\KeywordTok{getwd}\NormalTok{()}
\end{Highlighting}
\end{Shaded}

\begin{verbatim}
## [1] "D:/LND/COURSERA_DATA_SCIENCE/COURSERA_05_Reproducible Research/WEEK2_05RR_Markdown_knitr/Assignment"
\end{verbatim}

\begin{Shaded}
\begin{Highlighting}[]
\KeywordTok{setwd}\NormalTok{ (}\StringTok{"D:/LND/COURSERA_DATA_SCIENCE/COURSERA_05_Reproducible Research/WEEK2_05RR_Markdown_knitr/Assignment"}\NormalTok{)}
\end{Highlighting}
\end{Shaded}

Loading and preprocessing the data

install.packages(``ggplot2'') install.packages(``dplyr'')
install.packages(``chron'')

\begin{Shaded}
\begin{Highlighting}[]
\KeywordTok{library}\NormalTok{(ggplot2)}
\KeywordTok{library}\NormalTok{(dplyr)}
\end{Highlighting}
\end{Shaded}

\begin{verbatim}
## 
## Attaching package: 'dplyr'
\end{verbatim}

\begin{verbatim}
## The following objects are masked from 'package:stats':
## 
##     filter, lag
\end{verbatim}

\begin{verbatim}
## The following objects are masked from 'package:base':
## 
##     intersect, setdiff, setequal, union
\end{verbatim}

\begin{Shaded}
\begin{Highlighting}[]
\KeywordTok{library}\NormalTok{ (chron)}
\end{Highlighting}
\end{Shaded}

\subparagraph{1. Load the data
(i.e.~read.csv())}\label{load-the-data-i.e.read.csv}

\subparagraph{Downloading zip file if it doesn't already exist in the
workspace}\label{downloading-zip-file-if-it-doesnt-already-exist-in-the-workspace}

\begin{Shaded}
\begin{Highlighting}[]
\NormalTok{path <-}\StringTok{ }\KeywordTok{getwd}\NormalTok{()}
\KeywordTok{download.file}\NormalTok{(}\DataTypeTok{url =} \StringTok{"https://d396qusza40orc.cloudfront.net/repdata%2Fdata%2Factivity.zip"}
\NormalTok{              , }\DataTypeTok{destfile =} \KeywordTok{paste}\NormalTok{(path, }\StringTok{"dataFiles.zip"}\NormalTok{, }\DataTypeTok{sep =} \StringTok{"/"}\NormalTok{))}
\KeywordTok{unzip}\NormalTok{(}\DataTypeTok{zipfile =} \StringTok{"dataFiles.zip"}\NormalTok{)}
\end{Highlighting}
\end{Shaded}

Clear the workspace load raw activity data

\begin{Shaded}
\begin{Highlighting}[]
\KeywordTok{rm}\NormalTok{(}\DataTypeTok{list=}\KeywordTok{ls}\NormalTok{())}
\NormalTok{activity_raw <-}\StringTok{ }\KeywordTok{read.csv}\NormalTok{(}\StringTok{"activity.csv"}\NormalTok{, }\DataTypeTok{stringsAsFactors=}\OtherTok{FALSE}\NormalTok{)}
\end{Highlighting}
\end{Shaded}

\subsection{Process/transform the data suitable for
analysis}\label{processtransform-the-data-suitable-for-analysis}

\subsubsection{Transform the date attribute to an actual date
format}\label{transform-the-date-attribute-to-an-actual-date-format}

\begin{Shaded}
\begin{Highlighting}[]
\NormalTok{activity_raw}\OperatorTok{$}\NormalTok{date <-}\StringTok{ }\KeywordTok{as.POSIXct}\NormalTok{(activity_raw}\OperatorTok{$}\NormalTok{date, }\DataTypeTok{format=}\StringTok{"%Y-%m-%d"}\NormalTok{)}
\NormalTok{activity_raw <-}\StringTok{ }\KeywordTok{data.frame}\NormalTok{(}\DataTypeTok{date=}\NormalTok{activity_raw}\OperatorTok{$}\NormalTok{date, }
                           \DataTypeTok{weekday=}\KeywordTok{tolower}\NormalTok{(}\KeywordTok{weekdays}\NormalTok{(activity_raw}\OperatorTok{$}\NormalTok{date)), }
                           \DataTypeTok{steps=}\NormalTok{activity_raw}\OperatorTok{$}\NormalTok{steps, }
                           \DataTypeTok{interval=}\NormalTok{activity_raw}\OperatorTok{$}\NormalTok{interval)}
\end{Highlighting}
\end{Shaded}

Compute the day type (weekend or weekday)

\begin{Shaded}
\begin{Highlighting}[]
\NormalTok{activity_raw <-}\StringTok{ }\KeywordTok{cbind}\NormalTok{(activity_raw, }
                      \DataTypeTok{daytype=}\KeywordTok{ifelse}\NormalTok{(activity_raw}\OperatorTok{$}\NormalTok{weekday }\OperatorTok{==}\StringTok{ "saturday"} \OperatorTok{|}\StringTok{ }
\StringTok{                                       }\NormalTok{activity_raw}\OperatorTok{$}\NormalTok{weekday }\OperatorTok{==}\StringTok{ "sunday"}\NormalTok{, }\StringTok{"weekend"}\NormalTok{, }
                                     \StringTok{"weekday"}\NormalTok{))}

\NormalTok{activity <-}\StringTok{ }\KeywordTok{data.frame}\NormalTok{(}\DataTypeTok{date=}\NormalTok{activity_raw}\OperatorTok{$}\NormalTok{date, }
                       \DataTypeTok{weekday=}\NormalTok{activity_raw}\OperatorTok{$}\NormalTok{weekday, }
                       \DataTypeTok{daytype=}\NormalTok{activity_raw}\OperatorTok{$}\NormalTok{daytype, }
                       \DataTypeTok{interval=}\NormalTok{activity_raw}\OperatorTok{$}\NormalTok{interval,}
                       \DataTypeTok{steps=}\NormalTok{activity_raw}\OperatorTok{$}\NormalTok{steps)}

\KeywordTok{rm}\NormalTok{(activity_raw)}
\end{Highlighting}
\end{Shaded}

Checking activity frame

\begin{Shaded}
\begin{Highlighting}[]
\KeywordTok{dim}\NormalTok{(activity)}
\end{Highlighting}
\end{Shaded}

\begin{verbatim}
## [1] 17568     5
\end{verbatim}

\begin{Shaded}
\begin{Highlighting}[]
\KeywordTok{head}\NormalTok{(activity)}
\end{Highlighting}
\end{Shaded}

\begin{verbatim}
##         date weekday daytype interval steps
## 1 2012-10-01  monday weekday        0    NA
## 2 2012-10-01  monday weekday        5    NA
## 3 2012-10-01  monday weekday       10    NA
## 4 2012-10-01  monday weekday       15    NA
## 5 2012-10-01  monday weekday       20    NA
## 6 2012-10-01  monday weekday       25    NA
\end{verbatim}

\begin{Shaded}
\begin{Highlighting}[]
\KeywordTok{str}\NormalTok{(activity)}
\end{Highlighting}
\end{Shaded}

\begin{verbatim}
## 'data.frame':    17568 obs. of  5 variables:
##  $ date    : POSIXct, format: "2012-10-01" "2012-10-01" ...
##  $ weekday : Factor w/ 7 levels "friday","monday",..: 2 2 2 2 2 2 2 2 2 2 ...
##  $ daytype : Factor w/ 2 levels "weekday","weekend": 1 1 1 1 1 1 1 1 1 1 ...
##  $ interval: int  0 5 10 15 20 25 30 35 40 45 ...
##  $ steps   : int  NA NA NA NA NA NA NA NA NA NA ...
\end{verbatim}

\begin{Shaded}
\begin{Highlighting}[]
\KeywordTok{summary}\NormalTok{(activity)}
\end{Highlighting}
\end{Shaded}

\begin{verbatim}
##       date                 weekday        daytype         interval     
##  Min.   :2012-10-01   friday   :2592   weekday:12960   Min.   :   0.0  
##  1st Qu.:2012-10-16   monday   :2592   weekend: 4608   1st Qu.: 588.8  
##  Median :2012-10-31   saturday :2304                   Median :1177.5  
##  Mean   :2012-10-31   sunday   :2304                   Mean   :1177.5  
##  3rd Qu.:2012-11-15   thursday :2592                   3rd Qu.:1766.2  
##  Max.   :2012-11-30   tuesday  :2592                   Max.   :2355.0  
##                       wednesday:2592                                   
##      steps       
##  Min.   :  0.00  
##  1st Qu.:  0.00  
##  Median :  0.00  
##  Mean   : 37.38  
##  3rd Qu.: 12.00  
##  Max.   :806.00  
##  NA's   :2304
\end{verbatim}

\subparagraph{1. Make a histogram of the total number of steps taken
each
day}\label{make-a-histogram-of-the-total-number-of-steps-taken-each-day}

\begin{Shaded}
\begin{Highlighting}[]
\NormalTok{activity_total_steps <-}\StringTok{ }\KeywordTok{with}\NormalTok{(activity, }\KeywordTok{aggregate}\NormalTok{(steps, }\DataTypeTok{by =} \KeywordTok{list}\NormalTok{(date), }\DataTypeTok{FUN =}\NormalTok{ sum, }\DataTypeTok{na.rm =} \OtherTok{TRUE}\NormalTok{))}
\KeywordTok{names}\NormalTok{(activity_total_steps) <-}\StringTok{ }\KeywordTok{c}\NormalTok{(}\StringTok{"date"}\NormalTok{, }\StringTok{"steps"}\NormalTok{)}
\KeywordTok{hist}\NormalTok{(activity_total_steps}\OperatorTok{$}\NormalTok{steps, }\DataTypeTok{main =} \StringTok{"Total number of steps taken per day"}\NormalTok{, }\DataTypeTok{xlab =} \StringTok{"Total steps taken per day"}\NormalTok{,  }\DataTypeTok{col =} \StringTok{"lightblue"}\NormalTok{, }\DataTypeTok{ylim =} \KeywordTok{c}\NormalTok{(}\DecValTok{0}\NormalTok{,}\DecValTok{20}\NormalTok{), }\DataTypeTok{breaks =} \KeywordTok{seq}\NormalTok{(}\DecValTok{0}\NormalTok{,}\DecValTok{25000}\NormalTok{, }\DataTypeTok{by=}\DecValTok{2500}\NormalTok{), }\DataTypeTok{labels=}\OtherTok{TRUE}\NormalTok{)}
\KeywordTok{abline}\NormalTok{(}\DataTypeTok{v =} \KeywordTok{mean}\NormalTok{(activity_total_steps}\OperatorTok{$}\NormalTok{steps), }\DataTypeTok{lty =} \DecValTok{1}\NormalTok{, }\DataTypeTok{lwd =} \DecValTok{2}\NormalTok{, }\DataTypeTok{col =} \StringTok{"red"}\NormalTok{)}
\KeywordTok{abline}\NormalTok{(}\DataTypeTok{v =} \KeywordTok{median}\NormalTok{(activity_total_steps}\OperatorTok{$}\NormalTok{steps), }\DataTypeTok{lty =} \DecValTok{2}\NormalTok{, }\DataTypeTok{lwd =} \DecValTok{2}\NormalTok{, }\DataTypeTok{col =} \StringTok{"black"}\NormalTok{)}
\KeywordTok{legend}\NormalTok{(}\DataTypeTok{x =} \StringTok{"topright"}\NormalTok{, }\KeywordTok{c}\NormalTok{(}\StringTok{"Mean"}\NormalTok{, }\StringTok{"Median"}\NormalTok{), }\DataTypeTok{col =} \KeywordTok{c}\NormalTok{(}\StringTok{"red"}\NormalTok{, }\StringTok{"black"}\NormalTok{), }
       \DataTypeTok{lty =} \KeywordTok{c}\NormalTok{(}\DecValTok{1}\NormalTok{, }\DecValTok{2}\NormalTok{), }\DataTypeTok{lwd =} \KeywordTok{c}\NormalTok{(}\DecValTok{2}\NormalTok{, }\DecValTok{2}\NormalTok{))}
\end{Highlighting}
\end{Shaded}

\includegraphics{05RR_Course_Project1RPUB_files/figure-latex/unnamed-chunk-8-1.pdf}

\begin{Shaded}
\begin{Highlighting}[]
\NormalTok{##Mean}
\KeywordTok{mean}\NormalTok{(activity_total_steps}\OperatorTok{$}\NormalTok{steps)}
\end{Highlighting}
\end{Shaded}

\begin{verbatim}
## [1] 9354.23
\end{verbatim}

\begin{Shaded}
\begin{Highlighting}[]
\NormalTok{##Median}
\KeywordTok{median}\NormalTok{(activity_total_steps}\OperatorTok{$}\NormalTok{steps)}
\end{Highlighting}
\end{Shaded}

\begin{verbatim}
## [1] 10395
\end{verbatim}

\begin{Shaded}
\begin{Highlighting}[]
\KeywordTok{summary}\NormalTok{(activity_total_steps}\OperatorTok{$}\NormalTok{steps)}
\end{Highlighting}
\end{Shaded}

\begin{verbatim}
##    Min. 1st Qu.  Median    Mean 3rd Qu.    Max. 
##       0    6778   10395    9354   12811   21194
\end{verbatim}

\subsection{What is the average daily activity
pattern?}\label{what-is-the-average-daily-activity-pattern}

\subsection{\texorpdfstring{Excludes Missing Values``NA'' using
na.rm=TRUE}{Excludes Missing ValuesNA using na.rm=TRUE}}\label{excludes-missing-valuesna-using-na.rmtrue}

\begin{Shaded}
\begin{Highlighting}[]
\NormalTok{average_daily_activity <-}\StringTok{ }\KeywordTok{aggregate}\NormalTok{(activity}\OperatorTok{$}\NormalTok{steps, }\DataTypeTok{by=}\KeywordTok{list}\NormalTok{(activity}\OperatorTok{$}\NormalTok{interval), }\DataTypeTok{FUN=}\NormalTok{mean, }\DataTypeTok{na.rm=}\OtherTok{TRUE}\NormalTok{)}
\KeywordTok{names}\NormalTok{(average_daily_activity) <-}\StringTok{ }\KeywordTok{c}\NormalTok{(}\StringTok{"interval"}\NormalTok{, }\StringTok{"mean"}\NormalTok{)}

\KeywordTok{plot}\NormalTok{(average_daily_activity}\OperatorTok{$}\NormalTok{interval, average_daily_activity}\OperatorTok{$}\NormalTok{mean, }\DataTypeTok{type =} \StringTok{"l"}\NormalTok{, }\DataTypeTok{col=}\StringTok{"darkblue"}\NormalTok{, }\DataTypeTok{lwd =} \DecValTok{2}\NormalTok{, }\DataTypeTok{xlab=}\StringTok{"Interval"}\NormalTok{, }\DataTypeTok{ylab=}\StringTok{"Average number of steps"}\NormalTok{, }\DataTypeTok{main=}\StringTok{"Average number of steps per intervals"}\NormalTok{)}
\NormalTok{average_daily_activity[}\KeywordTok{which.max}\NormalTok{(average_daily_activity}\OperatorTok{$}\NormalTok{mean), ]}\OperatorTok{$}\NormalTok{interval}
\end{Highlighting}
\end{Shaded}

\begin{verbatim}
## [1] 835
\end{verbatim}

\begin{Shaded}
\begin{Highlighting}[]
\KeywordTok{abline}\NormalTok{(}\DataTypeTok{v =}\NormalTok{ average_daily_activity[}\KeywordTok{which.max}\NormalTok{(average_daily_activity}\OperatorTok{$}\NormalTok{mean), ]}\OperatorTok{$}\NormalTok{interval, }\DataTypeTok{lty =} \DecValTok{3}\NormalTok{, }\DataTypeTok{lwd =} \DecValTok{2}\NormalTok{, }\DataTypeTok{col =} \StringTok{"red"}\NormalTok{)}
\KeywordTok{legend}\NormalTok{(}\DataTypeTok{x =} \StringTok{"topright"}\NormalTok{, }\KeywordTok{c}\NormalTok{(}\StringTok{"Max(Avg Daily Actvity Mean)"}\NormalTok{), }\DataTypeTok{col =} \KeywordTok{c}\NormalTok{(}\StringTok{"red"}\NormalTok{),}\DataTypeTok{lty =} \KeywordTok{c}\NormalTok{(}\DecValTok{3}\NormalTok{), }\DataTypeTok{cex=}\FloatTok{0.65}\NormalTok{)}
\end{Highlighting}
\end{Shaded}

\includegraphics{05RR_Course_Project1RPUB_files/figure-latex/unnamed-chunk-10-1.pdf}

\begin{Shaded}
\begin{Highlighting}[]
\NormalTok{## Max Average }
\NormalTok{## average_daily_activity[which.max(average_daily_activity$mean), [1]}

\NormalTok{average_daily_activity[}\KeywordTok{which.max}\NormalTok{(average_daily_activity}\OperatorTok{$}\NormalTok{mean), ]}\OperatorTok{$}\NormalTok{interval}
\end{Highlighting}
\end{Shaded}

\begin{verbatim}
## [1] 835
\end{verbatim}

\begin{Shaded}
\begin{Highlighting}[]
\NormalTok{## Maximum Average Number of Steps}
\NormalTok{## average_daily_activity[which.max(average_daily_activity$mean), ][2]}

\NormalTok{average_daily_activity[}\KeywordTok{which.max}\NormalTok{(average_daily_activity}\OperatorTok{$}\NormalTok{mean), ]}\OperatorTok{$}\NormalTok{mean}
\end{Highlighting}
\end{Shaded}

\begin{verbatim}
## [1] 206.1698
\end{verbatim}

\section{Split into two sets: complete and
missing.}\label{split-into-two-sets-complete-and-missing.}

\begin{Shaded}
\begin{Highlighting}[]
\NormalTok{activity.missing <-}\StringTok{ }\NormalTok{activity[}\KeywordTok{is.na}\NormalTok{(activity}\OperatorTok{$}\NormalTok{steps),]}
\NormalTok{activity.complete<-activity[}\KeywordTok{complete.cases}\NormalTok{(activity),]}

\NormalTok{NA_count <-}\StringTok{ }\KeywordTok{sum}\NormalTok{(}\KeywordTok{is.na}\NormalTok{(activity}\OperatorTok{$}\NormalTok{steps))}
\NormalTok{NA_pos <-}\StringTok{ }\KeywordTok{which}\NormalTok{(}\KeywordTok{is.na}\NormalTok{(activity}\OperatorTok{$}\NormalTok{steps))}
\NormalTok{mean_vec <-}\StringTok{ }\KeywordTok{rep}\NormalTok{(}\KeywordTok{mean}\NormalTok{(activity}\OperatorTok{$}\NormalTok{steps, }\DataTypeTok{na.rm=}\OtherTok{TRUE}\NormalTok{), }\DataTypeTok{times=}\KeywordTok{length}\NormalTok{(NA_pos))}
\NormalTok{activity.complete[NA_pos, }\StringTok{"steps"}\NormalTok{] <-}\StringTok{ }\NormalTok{mean_vec}
\KeywordTok{head}\NormalTok{(activity.complete)}
\end{Highlighting}
\end{Shaded}

\begin{verbatim}
##           date weekday daytype interval   steps
## 289 2012-10-02 tuesday weekday        0 37.3826
## 290 2012-10-02 tuesday weekday        5 37.3826
## 291 2012-10-02 tuesday weekday       10 37.3826
## 292 2012-10-02 tuesday weekday       15 37.3826
## 293 2012-10-02 tuesday weekday       20 37.3826
## 294 2012-10-02 tuesday weekday       25 37.3826
\end{verbatim}

\section{Compute the total number of steps each day (NA values
removed)}\label{compute-the-total-number-of-steps-each-day-na-values-removed}

\begin{Shaded}
\begin{Highlighting}[]
\NormalTok{sum_data <-}\StringTok{ }\KeywordTok{aggregate}\NormalTok{(activity.complete}\OperatorTok{$}\NormalTok{steps, }\DataTypeTok{by=}\KeywordTok{list}\NormalTok{(activity.complete}\OperatorTok{$}\NormalTok{date), }\DataTypeTok{FUN=}\NormalTok{sum)}

\NormalTok{## Rename the attributes}
\KeywordTok{names}\NormalTok{(sum_data) <-}\StringTok{ }\KeywordTok{c}\NormalTok{(}\StringTok{"date"}\NormalTok{, }\StringTok{"total"}\NormalTok{)}
\end{Highlighting}
\end{Shaded}

\section{Compute the histogram of the total number of steps each
day}\label{compute-the-histogram-of-the-total-number-of-steps-each-day}

\begin{Shaded}
\begin{Highlighting}[]
\KeywordTok{hist}\NormalTok{(sum_data}\OperatorTok{$}\NormalTok{total, }
     \DataTypeTok{breaks=}\KeywordTok{seq}\NormalTok{(}\DataTypeTok{from=}\DecValTok{0}\NormalTok{, }\DataTypeTok{to=}\DecValTok{25000}\NormalTok{, }\DataTypeTok{by=}\DecValTok{2500}\NormalTok{),}
     \DataTypeTok{col=}\StringTok{"orange"}\NormalTok{, }
     \DataTypeTok{xlab=}\StringTok{"Total number of steps"}\NormalTok{, }
     \DataTypeTok{ylim=}\KeywordTok{c}\NormalTok{(}\DecValTok{0}\NormalTok{, }\DecValTok{30}\NormalTok{), }
     \DataTypeTok{main=}\StringTok{"Histogram of the total number of steps taken each day}\CharTok{\textbackslash{}n}\StringTok{(With missing data imputed}\CharTok{\textbackslash{}n}\StringTok{ NA Replaced by Mean value)"}\NormalTok{)}
\end{Highlighting}
\end{Shaded}

\includegraphics{05RR_Course_Project1RPUB_files/figure-latex/unnamed-chunk-13-1.pdf}

\begin{Shaded}
\begin{Highlighting}[]
\NormalTok{## Mean}
 \KeywordTok{mean}\NormalTok{(sum_data}\OperatorTok{$}\NormalTok{total)}
\end{Highlighting}
\end{Shaded}

\begin{verbatim}
## [1] 11126.8
\end{verbatim}

\begin{Shaded}
\begin{Highlighting}[]
\NormalTok{## Median}
\KeywordTok{median}\NormalTok{(sum_data}\OperatorTok{$}\NormalTok{total)}
\end{Highlighting}
\end{Shaded}

\begin{verbatim}
## [1] 10766.19
\end{verbatim}

\begin{Shaded}
\begin{Highlighting}[]
\NormalTok{## Clear the workspace}
\KeywordTok{rm}\NormalTok{(sum_data)}

\NormalTok{## Load the lattice graphical library---}
\KeywordTok{library}\NormalTok{(lattice)}
\end{Highlighting}
\end{Shaded}

\section{Compute the average number of steps taken, averaged across all
daytype
variable}\label{compute-the-average-number-of-steps-taken-averaged-across-all-daytype-variable}

\begin{Shaded}
\begin{Highlighting}[]
\KeywordTok{head}\NormalTok{(activity.complete)}
\end{Highlighting}
\end{Shaded}

\begin{verbatim}
##           date weekday daytype interval   steps
## 289 2012-10-02 tuesday weekday        0 37.3826
## 290 2012-10-02 tuesday weekday        5 37.3826
## 291 2012-10-02 tuesday weekday       10 37.3826
## 292 2012-10-02 tuesday weekday       15 37.3826
## 293 2012-10-02 tuesday weekday       20 37.3826
## 294 2012-10-02 tuesday weekday       25 37.3826
\end{verbatim}

\begin{Shaded}
\begin{Highlighting}[]
\NormalTok{activity.complete.daytype <-}\StringTok{ }\KeywordTok{aggregate}\NormalTok{(steps }\OperatorTok{~}\StringTok{ }\NormalTok{daytype}\OperatorTok{+}\NormalTok{interval, }\DataTypeTok{data=}\NormalTok{activity.complete, }\DataTypeTok{FUN=}\NormalTok{mean)}
\KeywordTok{head}\NormalTok{(activity.complete.daytype)}
\end{Highlighting}
\end{Shaded}

\begin{verbatim}
##   daytype interval    steps
## 1 weekday        0 7.212708
## 2 weekend        0 2.670186
## 3 weekday        5 5.751169
## 4 weekend        5 2.670186
## 5 weekday       10 5.751169
## 6 weekend       10 2.670186
\end{verbatim}

\section{Compute the time serie plot}\label{compute-the-time-serie-plot}

\begin{Shaded}
\begin{Highlighting}[]
\KeywordTok{xyplot}\NormalTok{(steps }\OperatorTok{~}\StringTok{ }\NormalTok{interval }\OperatorTok{|}\StringTok{ }\NormalTok{daytype, activity.complete.daytype, }
       \DataTypeTok{type=}\StringTok{"l"}\NormalTok{, }
       \DataTypeTok{lwd=}\DecValTok{1}\NormalTok{, }
       \DataTypeTok{xlab=}\StringTok{"Interval"}\NormalTok{, }
       \DataTypeTok{ylab=}\StringTok{"Mean Number of steps"}\NormalTok{, }
       \DataTypeTok{layout=}\KeywordTok{c}\NormalTok{(}\DecValTok{1}\NormalTok{,}\DecValTok{2}\NormalTok{))}
\end{Highlighting}
\end{Shaded}

\includegraphics{05RR_Course_Project1RPUB_files/figure-latex/unnamed-chunk-16-1.pdf}

\begin{Shaded}
\begin{Highlighting}[]
\NormalTok{## It seems that the weekday activities starts earlier than the weekends and weekday activities starts around 5-6am and weekend activities starts around 8am. }
\NormalTok{## Another observation is that from 10am to 5pm in the weekends have higher activity levels than the weekdays.}
\end{Highlighting}
\end{Shaded}


\end{document}
